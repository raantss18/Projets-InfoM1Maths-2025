\documentclass{beamer}

\usepackage[orientation=portrait,size=a4]{beamerposter}
\usepackage{amsmath,amssymb,amsthm}
\usepackage{xcolor}

\definecolor{bleunuit}{RGB}{25,25,112}
\definecolor{bleuclair}{RGB}{173,216,230}

\title{\textcolor{bleunuit}{Quelques Mathematiciens célèbre}}

\setbeamercolor{block title}{bg=bleunuit,fg=white}
\setbeamercolor{block body}{bg=bleuclair,fg=black}
\setbeamerfont{block title}{size=\Large}
\setbeamerfont{block body}{size=\Large}

\title{\textbf{\Huge Quelques Mathématiciens célèbres}}
\date{}

\begin{document}
 
 \begin{frame}[fragile]
  \maketitle
   \begin{columns}[T]
    \begin{column}{.50\textwidth}
     \begin{block}{Leonhard Euler\\(1707-1783)}
      \begin{itemize}
       \item Euler est un mathématicien et physicien suisse, qui passa la plus grande partie de sa vie dans l'Empire russe et en Allemagne. Il était notamment membre de l'Académie royale des sciences de Prusse à Berlin.
       \item Euler fit d'importantes découvertes dans des domaines aussi variés que le calcul infinitésimal et la théorie des graphes. Il introduisit également une grande partie de la terminologie et de la notation des mathématiques modernes, en particulier pour l'analyse mathématiques, comme la notion de fonction mathématiques. Il est aussi connu pour ses travaux en mécanique, en dynamique des fluides, en optique et en astronomie ou en géométrie du triangle. 
       \item \textit{Exemple de sa contribution en Mathématiques:} Formule d'Euler, pour tout reel \(x\),
   \[
      e^{ix} = \cos x + i\sin x
   \]      
     \end{itemize} 
   \end{block}
  \begin{block}{Bernhard Riemann\\(1826-1866)}
   \begin{itemize}
    \item Georg Friedrich Bernhard Riemann est un mathématicien allemand.
    \item Influent sur le plan théorique, il a apporté de nombreuses contributions importantes à la topologie, l'analyse, la géométrie différentielle et au calcul, certaines d'entre elles ayant permis par la suite le développement de la relativité générale.
    \item \textit{Exemple de sa contribution en Mathématiques:} Integrale de Riemann:\\
  Soit \( f \) une fonction définie sur l'intervalle \( [a, b] \). Si \( f \) est intégrable selon Riemann sur cet intervalle, alors l'intégrale de \( f \) sur \( [a, b] \) est donnée par :

\[
\int_a^b f(x) \, dx = \lim_{\| P \| \to 0} \sum_{i=1}^{n} f(x_i^*) \Delta x_i
\]
    \end{itemize}
   \end{block} 
 \end{column}
 \begin{column}{.50\textwidth}
   \begin{block}{Srinivasa Ramanujan \\ (1887-1920)}
      \begin{itemize}
       \item Ramanujan est un mathématicien indien. Issu d'une famille modeste de brahmanes orthodoxes, il est autodidacte, faisant toujours preuve d'une pensée indépendante et originale. Il apprend seul les mathématiques à partir de deux livres qu'il s'est procurés avant l'âge de seize ans.
       \item Il a etablit une grande quantité de résultats sur la théorie des nombres, sur les fractions continues et sur les séries divergentes, tandis qu'il crée son propre système de notation.
       \item \textit{Exemple de sa contribution en Mathématiques:} Ramanujan avait découvertes un ensemble de formule en 1910 et qui figuraient dans son premier article publié en Angleterre (sans aucune démonstration, et avec seulement quelques vagues indications sur leur origine),dont la plus surprenante (et d'ailleurs la plus efficace) est : 
 \[
   \frac{1}{\pi} = \frac{2\sqrt{2}}{9801} \sum_{k=0}^{\infty} \frac{(4k)! (1103+ 26390k)}{(k!)^4 396^{4k}}
 \]  
      \end{itemize}
     \end{block}
     \begin{block}{Augustin Louis Cauchy \\ (1789-1857)}
       \begin{itemize}
        \item Augustin Louis, baron Cauchy, est un mathématicien français, membre de l’Académie des sciences et professeur à l’École polytechnique.
        \item Ses recherches couvrent l’ensemble des domaines mathématiques de l’époque. On lui doit notamment en analyse l’introduction des fonctions holomorphes et des critères de convergence des suites et des séries entières. Ses travaux sur les permutations sont précurseurs de la théorie des groupes. En optique, on lui doit des travaux sur la propagation des ondes électromagnétiques.
        \item \textit{Exemple de sa contribution en Mathématiques:} Lagrange avait démontré que la résolution d’une équation algébrique générale de degré n passe par l’introduction d’une équation intermédiaire : sa résolvante dont le degré est le nombre de fonctions à n  variables obtenues par permutation des variables dans l’expression d’une fonction polynomiale.En 1813, Cauchy améliore cette estimation et démontre que ce nombre est supérieur au plus petit diviseur premier de n . Son résultat fut généralisé ensuite en l’actuel théorème de Cauchy.  
         \end{itemize}            
       \end{block}
     \end{column}
    \end{columns}
    N\begin{block}{Emmy Noether\\(1882-1935)}
      \begin{itemize}
       \item Emmy Noether est une mathématicienne allemande spécialiste d'algèbre abstraite et de physique théorique.
       \item Considérée par Albert Einstein comme le génie mathématique créatif le plus considérable produit depuis que les femmes ont eu accès aux études supérieures, elle a révolutionné les théories des anneaux, des corps et des algèbres. En physique, le théorème de Noether explique le lien fondamental entre la symétrie et les lois de conservation et est considéré comme aussi important que la théorie de la relativité
       \item \textit{Exemple de sa contribution en Mathématiques:}  Un important travail sur les nombres hypercomplexes et sur les représentations de groupes avait été accompli au dix-neuvième siècle et au début du vingtième, mais les résultats obtenus restaient disparates. Emmy Noether unifia ces résultats et construisit la première théorie générale des représentations des groupes et des algèbres. Elle rassembla la théorie de la structure des algèbres associatives et celle de la représentation des groupes en une seule théorie arithmétique des modules et des idéaux d'anneaux satisfaisant à des conditions de chaîne ascendante. Ce travail de Noether, à lui seul, s'avéra d'une importance fondamentale pour le développement de l'algèbre moderne
      \end{itemize}
    \end{block}
  
  \end{frame}
\end{document}
