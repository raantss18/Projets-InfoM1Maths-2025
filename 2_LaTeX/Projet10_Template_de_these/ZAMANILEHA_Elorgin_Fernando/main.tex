\documentclass[a4paper,12pt]{report}
\usepackage[utf8]{inputenc}
\usepackage[T1]{fontenc}
\usepackage{lmodern}
\usepackage[french]{babel}
\usepackage{geometry}
\geometry{top=2.5cm, bottom=2.5cm, left=2.54cm, right=2.54cm}
\usepackage{setspace}
\usepackage{titling}
\usepackage{tocloft}
\usepackage{natbib}
\usepackage{hyperref}
\hypersetup{
    colorlinks=true,
    linkcolor=black,
    citecolor=black,
    urlcolor=black
}
\usepackage{amsmath,amssymb}
\usepackage{graphicx}
\usepackage{xcolor}

% Configuration de la page de garde
\newcommand{\thesisTitle}{Titre de la Thèse}
\newcommand{\authorName}{Votre Nom}
\newcommand{\degree}{Docteur en [Votre Discipline]}
\newcommand{\university}{Votre Université}
\newcommand{\department}{Département de [Votre Département]}
\newcommand{\submissionDate}{Mai 2025}
\newcommand{\supervisor}{Pr. Nom du Directeur}
\newcommand{\juryPresident}{Pr. Nom du Président}
\newcommand{\juryMemberOne}{Dr. Nom du Membre 1}
\newcommand{\juryMemberTwo}{Dr. Nom du Membre 2}

% Configuration de la table des matières
\setlength{\cftbeforechapskip}{0.5em}
\renewcommand{\cftchapfont}{\normalfont\bfseries}
\renewcommand{\cftsecfont}{\normalfont}

% Configuration de la bibliographie
\bibliographystyle{plainnat}

\begin{document}

% Page de couverture
\begin{titlepage}
    \centering
    \vspace*{2cm}
    {\LARGE\bfseries \thesisTitle \par}
    \vspace{1.5cm}
    {\large Thèse présentée pour obtenir le grade de \par}
    {\Large \degree \par}
    \vspace{1cm}
    {\large par \par}
    {\Large \authorName \par}
    \vspace{1.5cm}
    {\large \university \par}
    {\large \department \par}
    \vspace{1cm}
    {\large Soutenue le \submissionDate \par}
    \vspace{1cm}
    {\large Directeur de thèse : \supervisor \par}
    \vspace{0.5cm}
    {\large Jury : \par}
    {\large \juryPresident, Président \par}
    {\large \juryMemberOne, Rapporteur \par}
    {\large \juryMemberTwo, Examinateur \par}
    \vfill
    {\large \includegraphics[width=0.3\textwidth]{logo-univ.png} \par}
\end{titlepage}

% Page de garde (vierge pour annotations du jury)
\clearpage
\thispagestyle{empty}
\null
\clearpage

% Page de titre (similaire à la couverture)
\begin{titlepage}
    \centering
    \vspace*{2cm}
    {\LARGE\bfseries \thesisTitle \par}
    \vspace{1.5cm}
    {\large Thèse présentée pour obtenir le grade de \par}
    {\Large \degree \par}
    \vspace{1cm}
    {\large par \par}
    {\Large \authorName \par}
    \vspace{1.5cm}
    {\large \university \par}
    {\large \department \par}
    \vspace{1cm}
    {\large Soutenue le \submissionDate \par}
    \vspace{1cm}
    {\large Directeur de thèse : \supervisor \par}
\end{titlepage}

% Table des matières
\clearpage
\tableofcontents
\clearpage

% Exemple de chapitre
\chapter{Introduction}
\onehalfspacing
Ceci est le chapitre d'introduction de votre thèse. Vous pouvez y décrire le contexte, les objectifs, et la structure de votre travail. Les thèses des universités comme Stanford et MIT privilégient une présentation claire et concise.

\section{Contexte}
Lorem ipsum dolor sit amet, consectetur adipiscing elit. Sed do eiusmod tempor incididunt ut labore et dolore magna aliqua.

\section{Objectifs}
Ut enim ad minim veniam, quis nostrud exercitation ullamco laboris nisi ut aliquip ex ea commodo consequat.

% Exemple de chapitre supplémentaire
\chapter{Méthodologie}
\onehalfspacing
Ce chapitre détaille les méthodes utilisées dans la recherche. Vous pouvez inclure des équations, des figures, ou des tableaux pour illustrer vos propos.

\section{Approche Expérimentale}
Duis aute irure dolor in reprehenderit in voluptate velit esse cillum dolore eu fugiat nulla pariatur.

% Bibliographie
\clearpage
\bibliography{references}
\begin{thebibliography}{9}
\bibitem{knuth1997}
Knuth, D. E. (1997). \emph{The Art of Computer Programming, Volume 1: Fundamental Algorithms}. Addison-Wesley.
\bibitem{lamport1994}
Lamport, L. (1994). \emph{\LaTeX: A Document Preparation System}. Addison-Wesley.
\end{thebibliography}

\end{document}