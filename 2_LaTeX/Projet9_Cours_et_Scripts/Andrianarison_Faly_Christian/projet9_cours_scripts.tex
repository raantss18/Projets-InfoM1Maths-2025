\documentclass[a4paper,12pt]{article}
\usepackage[utf8]{inputenc}
\usepackage[a4paper,margin=2cm]{geometry}
\usepackage[french]{babel}
\usepackage[T1]{fontenc}
\usepackage{tgadventor}
\renewcommand{\familydefault}{\sfdefault}
\usepackage{amsmath, amssymb}
\usepackage{listings}
\setlength{\parindent}{0pt}
\newtheorem{theorem}{Théorème}
\usepackage{mathpazo}


\title{Zéros des fonctions}

\begin{document}

\begin{center}
        \Huge \textbf{ZEROS DES FONCTIONS}
\end{center}
\vspace{1cm}

Dans ce chapitre nous allons appliquer toutes les notions précédentes sur les suites et les fonctions, à la recherche des zéros des fonctions. Plus précisément, nous allons voir trois méthodes afin de trouver des approximations des solutions d'une équation du type $ ( f (x) = 0) $.

\section{La dichotomie}
\subsection{Pricipe de la dichotomie}
    Le principe de dichotomie repose sur la version suivante du \textit{théorème des valeurs intermédiaires} :
    
    \begin{theorem}
    Soit $ f:[a,b] \longrightarrow \mathbb{R}$ une fonction continue sur un segment.\\
    Si $f(a) \cdot f(b) \leq 0$, alors il existe $l \in [a, b] $ tel que  $f(l) = 0$.
    \end{theorem}
    
    La condition $f(a) \cdot f(b) \leq 0$ signifie que $f(a)$ et $f(b)$ sont de signes opposés (ou que l'un des deux est nul). L'hypothèse de continuité est essentielle !

    \vspace{0.5cm}

    Ce théorème affirme qu'il existe au moins une solution de l'équation $f (x) = 0$ dans l'intervalle $[a, b]$.
    Pour le rendre effectif, et trouver une solution (approchée) de l'équation $ f(x) = 0$, il s'agit maintenant de l'appliquer sur un intervalle suffisamment petit. On va voir que cela permet d'obtenir un solution de
    l'équation $f (x) = 0$ comme la limite d'une suite.

    \vspace{0.5cm}
    
    Voici comment construire une suite d'intervalles emboîtés, dont la longueur tend vers $0$, et contenant chacun une solution de l'équation $f (x) = 0$.
    On part d'une fonction $f : [a, b] \rightarrow \mathbb{R}$ continue, avec $a < b$, et $f (a)\cdot f (b) \leq 0$.
    Voici la première étape de la construction : on regarde le signe de la valeur de la fonction $f$ appliquée au
    point milieu $\frac{a+b}{2}$.
    \begin{itemize}
        \item Si $f(a) \cdot f(\frac{a+b}{2}) \leq 0$, alors il existe $c \in [a, \frac{a+b}{2}]$ tel que $f(c) = 0$. 
        \item Si $f(a) \cdot f(\frac{a+b}{2}) > 0$, cela implique que $f ( \frac{a+b}{2}) \cdot f(b) \leq 0$, et alors il existe $c \in [\frac{a+b}{2}, b]$ tel que $f (c) = 0$.
    \end{itemize} 

    \vspace{0.5cm}
    
    Nous avons obtenu un intervalle de longueur moitié dans lequel l'équation $f (x) = 0$ admet une solution.
    On itère alors le procédé pour diviser de nouveau l'intervalle en deux.
    Voici le processus complet : \\
    \begin{itemize}
        \item \textbf{Au rang 0: } \\ On pose $a_0 = a$, $b_0 = b$. Il existe une solution $x_0$ de l'équation $f (x) = 0$ dans l'intervalle $[a_0, b_0]$.
        
        \item \textbf{Au rang 1: }
            \begin{itemize}
                \item[-] Si $f(a_0) \cdot f(\frac{a_0+b_0}{2}) \leq 0$, alors on pose $a_1 = a_0$ et $b_1 = \frac{a_0+b_0}{2}$,
                \item[-] Sinon on pose $a_ = \frac{a_0 + b_O}{2}$ et $b_1 = b.$
                \item[-] Dans les deux cas, il existe une solution $x_1$ de l'équation $f (x) = 0$ dans l'intervalle $[a_1, b_1]$.
            \end{itemize}
            
        \item \dots
        
        \item \textbf{Au rang n:}  supposons construit un intervalle $[a_n,b_n]$, de longueur $\frac{b-a}{2^n}$, et contenant une solution $x_n$ de l'équation $f (x) = 0$. Alors :
            \begin{itemize}
                \item[-] Si $f(a_n) \cdot f(\frac{a_n+b_n}{2}) \leq 0$, alors on pose $a_{n+1} = a_n$ et $b_{n+1} = \frac{a_n + b_n}{2}$,
                \item[-] sinon on pose $a_{n+1} = \frac{a_n + b_n}{2}$ et $b_{n+1} = b_n$.
                \item[-] Dans les deux cas, il existe une solution $x_{n+1}$ de l'équation $f (x) = 0$ dans l'intervalle $[a_{n+1}, b_{n+1}]$.
            \end{itemize}
    \end{itemize}

    \vspace{0.5cm}
    
    À chaque étape on a $$ a_n \leq xn \leq b_n $$.
    On arrête le processus dès que $b_n - a_n = \frac{b - a}{2^n}$ est inférieur à la précision souhaitée.

    \vspace{0.5cm}
    
    Comme $(a_n)$ est par construction une suite croissante, $(b_n)$ une suite décroissante, et $(b_n - a_n) \rightarrow 0$ lorsque $n \rightarrow +\infty$, les suites $(a_n)$ et $(b_n)$ sont adjacentes et donc elles admettent une même limite. D'après le théorème des gendarmes, c'est aussi la limite disons de la suite $(x_n)$. La continuité de $f$ montre que
    $f(l) = lim_{n \rightarrow +\infty}f(x_n) = lim_{n \rightarrow +\infty} 0 = 0$. Donc les suites $(a_n)$ et $(b_n)$ tendent toutes les deux vers $l$, qui est
    une solution de l'équation $f (x) = 0$.
    
\subsection{Résultat numérique pour $\sqrt{10}$}
    Nous allons calculer une approximation de $\sqrt{10}$. Soit la fonction $f$ définie par $f (x) = x^2 - 10$, c'est une fonction continue sur $\mathbb{R}$ qui s'annule en $\pm \sqrt{10}$ . De plus $\sqrt{10}$ est l'unique solution positive de l'équation $f (x) = 0$. Nous pouvons restreindre la fonction $f$ à l'intervalle $[3, 4]$ : en effet $3^2 = 9 \leq 10$ donc $3 \leq \sqrt{10}$ et $4^2 = 16 \geq 10$ donc $4 \geq \sqrt{10}$. En d'autre termes $f(3) \leq 0$ et $f(4) \geq 0$, donc l'équation $f (x) = 0$ admet une solution dans l'intervalle $[3, 4]$ d'après le théorème des valeurs intermédiaires, et par unicité c'est $\sqrt{10}$, donc $ \sqrt{10} \in [3, 4]$.
    Notez que l'on ne choisit pas pour $f$ la fonction $x \mapsto x - \sqrt{1}$ car on ne connaît pas la valeur de $\sqrt{10}$ . C'est ce que l'on cherche à calculer !

    \vspace{0.5cm}
    
    Voici les toutes premières étapes:
    \begin{enumerate}
        \item On pose $a_0 = 3$ et $b_0 = 4$, on a bien $f(a_0) \leq 0$ et $f(b_0) \geq 0$. On calcule $\frac{a_0 + b_0}{2} = 3.5$ puis $f(\frac{a_0 + b_0}{2})$: $f(3.5) = 3.5^2 - 10 = 2.25 \geq 0$. Donc $\sqrt{10}$ est dans l'intervalle $[3, 3.5]$ et on pose $a_1 = a_0 = 3$ et $b_1 = \frac{a_0 + b_0}{2} = 3.5$.
        
        \item On sait donc que $f(a_1) \leq 0$ et $f(b_1) \geq 0$. On calucle $f(\frac{a_1+b_1}{2}) = f(3.25) = 0.5625 \geq 0$, on pose $a_2 = 3$ et $b_2 = 3.25$.
        
        \item On calcule $f(\frac{a_2 + b_2}{2}) = f(3.125) = -0.23\dots \leq 0$. Comme $f(b_2) \geq 0$ alors cette fois $f$ s'annule sur le second intervalle $[\frac{a_2+b_2}{2}, b_2]$ et on pose $a_3 = \frac{a_2+b_2}{2} = 3.125$ et $b_3 = b_2 = 3.25$.  
    \end{enumerate}

    A ce stade, on a prouvé: $3.125 \leq \sqrt{10} \leq 3.25$.
    Voici la suite des étapes: 
    \begin{center}
        \begin{tabular}{l l}
            $a_0 = 3$   &  $b_0 = 4$ \\
            $a_1 = 3$   &  $b_1 = 3.5$ \\
            $a_2 = 3$   &  $b_2 = 3.25$ \\
            $a_3 = 3.125$   &  $b_3 = 3.25$ \\
            $a_4= 3.125$   &  $b_4 =3.1875$ \\
            $a_5 = 3.15625$   &  $b_5 = 3.1875$ \\
            $a_6 = 3.15625$   &  $b_6 = 3.171875$ \\
        \end{tabular}    
    \end{center}
    
    Donc en 8 étapes on obtient l'encadrement:
    $$ 3.160 \leq \sqrt{10} \leq 3.165 $$

    En particulier, on vient d'obtenir les deux premières décimales: $\sqrt{10} = 3.16 \dots$

\subsection{Résultats numériques pour $(1.10)^{\frac{1}{12}}$}
    Nous cherchons maintenant une approximation de $(1.10)^{\frac{1}{12}}$. Soit $f(x) = x^12 - 1.10 $. On pose $a_0 = 1$ et $b_0 = 1.1$. Alors $f(a_0) = -0.10 \leq 0$ et $f(b_0) = 2.038 \dots \geq 0$.

    \begin{center}
        \begin{tabular}{l l}
            $a_0 = 1$   &  $b_0 = 1.10$ \\
            $a_1 = 1$   &  $b_1 = 1.05$ \\
            $a_2 = 1$   &  $b_2 = 1.025$ \\
            $a_3 = 1$   &  $b_3 = 1.0125$ \\
            $a_4 = 1.00625$   &  $b_4 =1.0125$ \\
            $a_5 = 1.00625$   &  $b_5 = 1.00937$ \dots\\
            $a_6 = 1.00781 \dots$   &  $b_6 = 1.00937$ \dots\\
            $a_7 = 1.00781 \dots$   &  $b_6 = 1.00937$ \dots\\
            $a_8 = 1.00781 \dots$   &  $b_6 = 1.00820$ \dots\\
        \end{tabular}
    \end{center}

    Donc en 8 étapes on obtient l'encadrement:
    $$ 1.00781 \leq (1.10)^{\frac{1}{12}} \leq 1.00821 $$

\subsection{Calcul de l'erreur}
    La méthode de dichotomie a l'énorme avantage de fournir un encadrement d'une solution $l$ de l'équation ($f(x) = 0$). Il est dinc facile d'avoir une majoration de l'erreur. En effet, à chaque étape, la taille l'intervalle contenant $l$ est divisée par 2. Au départ, on sait que $l \in [a, b]$ (de longueur $b-a$); puis $l \in [a_1, b_1]$ (de longueur $\frac{a-b}{2}$); puis $l \in [a_2, b_2]$ (de longueur $\frac{b-a}{4}$); \dots; $[a_n, b_n]$ étant de longueur $\frac{b-a}{2^n}$.

    \vspace{0.5cm}
    
    Si, par exemple, on souhaite obtenir une approximation de $l$ à $10^{-N}$ près, comme on sait que $a_n \leq l \leq b_n$, on obtient $|l - a_n| \leq |b_n - a_n| = \frac{b - a}{2^n}$. Donc pour avoir $|l - a_n| \leq 10^{-N}$, il suffit de choisir $n$ tel que $\frac{b-a}{2^n} \leq 10^{-N}$. Nous allons utiliser le logarithme décimal: 

    \vspace{0.5cm}
    
        \begin{tabular}{l l}
            $\frac{b-a}{2^n} \leq 10^{-N}$ & $\Leftrightarrow (b-a) \cdot 10^N \leq 2^n$ \\
                            & $ \Leftrightarrow \log(b-a) + \log(10^N) \leq \log(2^n)$ \\
                            & $ \Leftrightarrow \log(b-a) + N \leq n \log 2 $ \\
                            & $ \Leftrightarrow n \geq \frac{N + \log(b-a)}{\log 2} $\\
        \end{tabular} \\

    \vspace{0.5cm}
        
    Sachant $ \log 2 = 0.301 \dots$, si par exemple $b-a \leq 1$, voici le nombre d'itérations suffisantes pour avoir une précision de  $10^{-N}$ (ce qui correspond, à peu près, à $N$ chiffres exacts après la virgule).
    \begin{center}
        \begin{tabular}{l l}
            $10^{-10}$ ($~ 10$ décimales)   &  $34$ itérations \\
            $10^{-100}$ ($~100$ décimales)   &  $333$ itérations \\
            $10^{-100}$ ($~1000$ décimales)   &  $3322$ itérations \\
        \end{tabular}
    \end{center}
    
    Il faut entre 3 et 4 itérations supplémentaires pour obtenir une nouvelle décimale.

    \vspace{0.5cm}
    
    \textbf{Remarque.} \\
    En toute rigueur il ne faut pas confondre précision et nombre de décimales exactes, par exemple $0.999$ est une approximation de $1,000$ à $10^{-3}$ près, mais aucune décimale après la virgule n'est exacte. En pratique, c'est la précision qui est la plus importante, mais il est plus frappant de parler  de nombre de décimales exactes.

\subsection{Algorithmes}

    Voici comment implémenter la dichotomie dans le langage \texttt{Python}. Tout d'abord on définit une fonction $f$ (ici par exemple $f(x) = x^2 - 10$): \\
    \textbf{Code 37} (\textit{dichotomie.py (1)}).
    \begin{lstlisting}[language=Python]
        def f(x):
            return x*x - 10
    \end{lstlisting}

    \vspace{0.5cm}

    Puis la dichotomie proprement dite: en entrée de la focntion, on a pour variables $a$, $b$ et $n$ le nombre d'étapes voulues. \\
    \textbf{Code 38} (\textit{dichotomie.py (2)}).
    \begin{lstlisting}[language=Python]
    def dicho(a,b,n):
        for i in range(n):
            c = (a+b)/2
            if f(a)*f(c) <= 0:
                b = c 
            else:
                a = c
            return a,b
    \end{lstlisting}

    \vspace{0.5cm}

    Même algoritgme, mais avec cette fois en entrée la précision souhaitée: \\
    \textbf{Code 39} (\textit{dichotomie.py (3)}).
    \begin{lstlisting}[language=Python]
        def dichobis(a,b,prec):
            while b-a > prec:
                c = (a+b)/2
                if f(a)*f(c) <= 0:
                    b = c 
                else:
                    a = c
            return a,b
    \end{lstlisting}

    \vspace{0.5cm}

    Enfin, voici la version récursive de l'algorithme de dichotomie. \\
    \textbf{Code 40} (\textit{dichotomie.py (4)}).
    \begin{lstlisting}[language=Python]
        def dichotomie(a,b,prec):
            if b-a <= prec:
                return a, b
            else
                c = (a+b)/2    
                if f(a)*f(c) <= 0:
                    return dichotomie(a,c,prec)
                else:
                    return dichotomie(c,b,prec)
    \end{lstlisting}

\end{document}