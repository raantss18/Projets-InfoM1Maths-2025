% Préparation du document avec les packages nécessaires
\documentclass[a4paper,12pt]{article}
\usepackage[utf8]{inputenc}
\usepackage[T1]{fontenc}
\usepackage[french]{babel}
\usepackage{amsmath, amssymb}
\usepackage{geometry}
\geometry{margin=2cm}
% Configuration des polices (chargées en dernier)
\usepackage{mathptmx} % Police Times pour compatibilité

\begin{document}
	
	% Titre du document
	\title{Discussion des résultats : Marches sur un demi-plan}
	\author{ZAMANILEHA Elorgin Fernando}
	\date{14 Mai 2025}
	\maketitle
	
	% Introduction brève
	\section*{Introduction}
	On étudie les marches dans le demi-plan \( y \geq 0 \), partant de \( (0,0) \), avec pas \( \{\uparrow (0,1), \downarrow (0,-1), \rightarrow (1,0)\} \). On note \( a_n \) le nombre de marches de longueur \( n \) revenant à l'origine. Les résultats suivants ont été obtenus via un programme Python optimisé.
	
	% Résultats des termes a_n
	\section*{1. Termes \( a_n \)}
	Les 30 premiers termes \( a_n \) (pour \( n = 0 \) à \( 29 \)) sont :
	\[
	[1, 0, 1, 0, 2, 0, 5, 0, 14, 0, 42, 0, 132, 0, 429, 0, 1430, 0, 4862, 0, 16796, 0, 58786, 0, 208012, 0, 742900, 0, 2674440, 0]
	\]
	Les termes impairs sont nuls, car revenir à \( (0,0) \) nécessite un nombre pair de pas verticaux (\( \uparrow \) et \( \downarrow \)).
	
	% Série génératrice
	\section*{2. Série génératrice}
	La série génératrice jusqu'à \( x^{29} \) est :\\
	\(
	A(x) = 1 + x^2 + 2x^4 + 5x^6 + 14x^8 + 42x^{10} + 132x^{12} + 429x^{14} + 1430x^{16} + 4862x^{18} + 16796x^{20} + 58786x^{22} + 208012x^{24} + 742900x^{26} + 2674440x^{28}.
	\)\\
	Seuls les termes de degré pair apparaissent, cohérent avec les \( a_n \).
	
	% Approximation de Padé
	\section*{3. Approximation de Padé}
	L'approximation de Padé [14/14] donne :
	\begin{itemize}
		\item Numérateur : \( 1 - 13x^2 + 66x^4 - 165x^6 + 210x^8 - 126x^{10} + 28x^{12} - x^{14} \),
		\item Dénominateur : \( 1 - 14x^2 + 78x^4 - 220x^6 + 330x^8 - 252x^{10} + 84x^{12} - 8x^{14} \).
	\end{itemize}
	Le dénominateur suggère une singularité en \( x \approx 1/3 \), indiquant un rayon de convergence \( \rho = 3 \).
	
	% Vérification asymptotique
	\section*{4. Vérification asymptotique}
	On vérifie \( a_n \sim C \cdot 3^n \cdot n^{-3/2} \). La constante estimée est \( C \approx 0.450158 \), proche de \( \sqrt{2}/\pi \approx 0.450158 \), confirmant l'asymptotique. Un graphique montre la convergence de \( C \), avec une échelle linéaire et une annotation claire.
	
	% Discussion sur la D-finitude
	\section*{5. D-finitude}
	La série \( A(x) \) est probablement D-finie, car :
	\begin{itemize}
		\item Les marches restreintes ont des séries D-finies (Banderier \& Flajolet, 2002).
		\item Le dénominateur de Padé indique une équation différentielle linéaire.
		\item L'asymptotique \( n^{-3/2} \) suggère une singularité algébrique.
	\end{itemize}
	Une récurrence linéaire exacte pourrait être obtenue via l'algorithme LLL.
	
	% Conclusion
	\section*{Conclusion}
	Les résultats confirment les propriétés combinatoires des marches sur le demi-plan. La méthode itérative évite les débordements, et l'approximation de Padé révèle une structure D-finie, validée par l'asymptotique.
	
\end{document}